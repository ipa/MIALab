% !TeX spellcheck = en-GB
\section{Introduction}
\IEEEPARstart{P}{ersonal} femoral implants feature less attrition, therefore more comfortable wearing and lower implant related issues after implantation. Furthermore the operation can be carried out more swiftly, if the implant is pre-operational adjusted to the patient. The personalisation of femoral implant requires an accurate segmentation of the femur. With an upcoming awareness of radiation, we investigate in an algorithm to segment the femur in the knee joint on MRI images. To be applicable in clinical routine the method should be automatic and robust. Additionally we set our goal for the accuracy with a DICE of 0.95 and for the robustness with a standard deviation of 0.05.
%the trend of segmenting bones with MRI instead of CT data is increasing.

In this work we propose a method for fast and robust automatic femur segmentation from MRI data. The automatic segmentation is based on a Random Forest (RF) classifier \cite{rf} and does not require any manual interaction. In the following sections the segmentation pipeline, the training and testing of the segmentation and the performance of the segmentation are described and discussed.