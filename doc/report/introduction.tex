% !TeX spellcheck = en-GB
\section{Introduction}
\IEEEPARstart{P}{ersonal} femoral implants feature less attrition, therefore more comfortable wearing and lower implant related issues after installation. Furthermore the operation can be carried out more swiftly, if the implant is pre-operational adjusted to the patient. The personalisation of femoral implant requires an accurate segmentation of the femur. With an upcoming awareness of radiation, the trend of segmenting bones with MRI instead of CT data is increasing.
In this work we propose a fast and robust automatic femur segmentation from MRI data. The automatic segmentation is based on a Random Forest classifier and does not require any manual interaction. In the following sections the segmentation pipeline, the training and testing of the segmentation and the performance of the segmentation are described and discussed.

% Abstract:
% In this work we propose a fast and robust automatic femur segmentation from MRI data, in order to manufacture personal femoral implants.