% !TeX spellcheck = en-GB
\section{Discussion}
The results in \autoref{fig:dicecrossvalid} let us conclude, that we achieved our goal of a robust algorithm. % delete this one, make the reference in the next
With the achieved standard deviation and range of the results we can expect to get a useful result also with new data. Because the cross validation creates a new model for each of the 5 runs, we can deduce, that we do not have an over fitting of our Random Forest model.

In order to use the algorithm for femoral implant planning, mainly the condyles have to be segmented well. With the 3D volume rendering from the best case result in \autoref{fig:bestcase} it is possible to measure distances between important structures to plan a femoral implant. However, the epicondyles are not fully segmented, which is something that has to be improved in the future. In the case of \autoref{fig:worstcase} one could do the measurements needed, although there are lots false positives. However, the accuracy is probably not high enough for clinical routine.

% following paragraphs need rephrasing
The worst case result from \autoref{fig:worstcase} could be corrected by adjusting the post processing. This was not done, because by doing so, another segmentation would remove too much true positives. These false positives are not filtered by the post processing, because the connection to the true positives is too "thick". Therefore a bigger kernel for the morphological opening would have to be used.

As mentioned in the beginning, our main goal is to develop a robust algorithm. Therefore our approach to keep the algorithm simple turned out to be a good choice. It was also a good approach to remove all steps that worsened the results in some cases. As an example we removed the feature skewness, as it worsened the results slightly in half of the cases.