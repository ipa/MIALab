% !TeX spellcheck = en-GB
\section{Discussion}
The results in \autoref{fig:dicecrossvalid} with a standard deviation of $0.03$ let us conclude, that we achieved our goal of a robust algorithm, which we set with $0.05$ standard deviation. We validated our algorithm with a 5 fold cross validation to ensure the algorithm is tested on MRI volumes, that have not been used for training. Therefore we can expect it to perform with the same accuracy on new data, without overfitting of our Random Forest model. With a mean accuracy of $0.91$ DICE the goal of $0.95$ has not been reached. 

In order to use the algorithm for femoral implant planning, mainly the condyles have to be segmented well. With the 3D volume rendering from the best case result in Figure\autoref{fig:bestcase} it is possible to measure distances between important structures to plan a femoral implant. However, the epicondyles are not fully segmented, which is something that has to be improved in the future. In the case of Figure\autoref{fig:worstcase} one could perform the required measurements, although there are a lot of false positives. However, the over all segmentation in this case is not sufficient for clinical routine yet.

In the particular case of\autoref{fig:worstcase} the algorithm could have yielded better results with a more aggressive post processing approach. By applying the morphological opening with a bigger kernel the false positives could have been reduced. Because a bigger kernel would be able to split the connection from the bone to the artefacts. The separated artefacts are then discarded by keeping only the largest volume in the next post processing step. Nonetheless, this approach would also remove true positives in and over all lower the segmentation performance in other images. 

As mentioned in the beginning, our main goal is to develop a robust algorithm. This has been achieved by keeping the algorithm simple, which prevents overfitting and long segmentation time. Therefore all steps that worsened the results in some cases have been removed. As an example we did not include skewness as a feature, as it worsened the results slightly in more than half of the cases.