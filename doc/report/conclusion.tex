% !TeX spellcheck = en-GB
\section{Conclusion}
We developed a fast and robust automatic segmentation algorithm. The loading of one MRI volume until the final segmentation requires in average $2.5$ minutes on a commercial laptop. With a DICE of $0.91$ and a standard deviation of $0.03$ the segmentation has a sufficient robustness, but lacks in accuracy, which will be part of the future development.

In order to enhance the accuracy, we propose to include prior information of the patient. One could create several Random Forest models for a specific group of people, where the groups differ in age, sex, weight or height. Prior information could also be considered by supporting the segmentation with a Statistic Shape Model \cite{heimann2009statistical}. However, the performance of the Statistic Shape Model depends strongly on its initialization and may therefore decrease the robustness of the segmentation.

An other potential way to increase the accuracy is to exchange the 2D features of the Random Forest by 3D features. This might represent the data better, since an additional dimension is considered. The challenge is that the slice spacing is multiple times larger than the pixel spacing. The feature kernels must therefore be adjusted and this might lead to a loss of useful properties such as kernel separability.
